\documentclass[assignment = 3]{homework}

\usepackage{caption, subcaption, pdfpages, float}
\usepackage{graphics, wrapfig, pgf, graphicx}
\usepackage{enumitem}
\graphicspath{{../}}


% pacotes para importar código
\usepackage{caption, booktabs}
\usepackage[section, newfloat]{minted}
\definecolor{sepia}{RGB}{252,246,226}
\setminted{
    bgcolor = sepia,
    style   = pastie,
    frame   = leftline,
    autogobble,
    samepage,
    python3,
}
\setmintedinline{
    bgcolor={}
}

% ambientes de códigos de Python
\newmintedfile[pyinclude]{python}{}
\newmintinline[pyline]{python}{}
\newcommand{\pyref}[2]{\href{#1}{\texttt{#2}}}

% \SetupFloatingEnvironment{listing}{name=Código}
% \captionsetup[listing]{position=below,skip=-1pt}

\usepackage{csquotes}
\usepackage[style=verbose-ibid,autocite=footnote,notetype=foot+end,backend=biber]{biblatex}
\addbibresource{referencias.bib}
\usepackage[section]{placeins}

\usepackage[hidelinks]{hyperref}
\usepackage[noabbrev, nameinlink, brazilian]{cleveref}
\hypersetup{
    pdftitle  = {MC920 - Trabalho 3 - 187679},
    pdfauthor = {Tiago de Paula}
}

\newcommand{\textref}[2]{
    \hyperref[#2]{#1 \ref*{#2}}
}

\usepackage{import, multirow}
\usepackage{tikz}
\usetikzlibrary{matrix}
\usetikzlibrary{positioning}

\newenvironment{kmatrix}[1][1.3cm]{
    \begin{tikzpicture}[node distance=0cm]
        \tikzset{square matrix/.style={
                matrix of nodes,
                column sep=-\pgflinewidth, row sep=-\pgflinewidth,
                nodes={draw,
                    minimum height=#1,
                    anchor=center,
                    text width=#1,
                    align=center,
                    inner sep=0pt
                },
            },
            square matrix/.default=#1
        }
}{
    \end{tikzpicture}%
}

\newcommand*{\Scale}[2][4]{\scalebox{#1}{\ensuremath{#2}}}%

\newcommand{\red}[1]{\textcolor{red}{\textbf{#1}}}

\begin{document}

    \pagestyle{main}

    % \section{Introdução} \label{sec:intro}

Este trabalho teve como objetivo a implementação de técnicas de pontilhado com difusão de erros. A ideia é reduzir a quantidade de cores tentando manter a imagem o mais próximo possível da original. Isso é feito reduzindo a intesidade para seu limite mais próximo, ao longo de um caminho pela imagem, mas aplicando uma distribuição de erros na vizinhança do pixel.

A \cref{fig:base} apresenta as imagens base deste trabalho, usadas para análise e discussão das formas diferentes de aplicar o pontilhado. As figuras \ref{fig:baboon} e \ref{fig:peppers} são $512 \times 512$, enquanto a \cref{fig:monalisa} é $256 \times 256$ e a \cref{fig:watch} é $1024 \times 768$. Ao todo serão aplicadas 6 distribuições de erro com 4 curvas diferentes em cada figura. As distribuições de erros estão apresentadas na \cref{sec:distribuicoes}, equanto as curvas de varrimento podem ser encontradas na \cref{sec:varredura}.

% \begin{figure}[H]
%     \centering
%     \newcommand{\histograma}[2][width=\textwidth]{
    \begin{figure}[H]
        \centering
        \begin{subfigure}{0.35\textwidth}
            \centering
            \includegraphics[#1]{base/#2.png}
            \caption{Imagem.}
        \end{subfigure}%
        \begin{subfigure}{0.64\textwidth}
            \centering
            \import{../resultados/hist}{#2.pgf}
            \caption{Histograma.}
        \end{subfigure}

        \caption{\texttt{imagens/#2.pgm}}
        \label{fig:base:#2}
    \end{figure}
}

%     \caption{Imagens base da comparação dos filtros.}
%     \label{fig:base}
% \end{figure}


    % \section{O Programa}

Além das bibliotecas padrão de Python, foram utilizados os pacotes NumPy \autocite{ref:numpy} e, de forma opcional, Numba \autocite{ref:numba}.

\subsection{Código Fonte}

    O programa foi desenvolvido em Python 3.8, mas deveria funcionar com a versões 3.7 e 3.9 também. Além disso, o código fonte foi separado nos seguintes arquivos:

    \begin{description}
        \item[main.py] É o corpo do programa, resposável por processar os comandos e as opções da linha de comando.

        \item[lib] Pacote interno com as operações de pontilhado.

        \begin{description}
            \item[lib/\_\_init\_\_.py] Funções gerais de aplicação do pontilhado em imagens coloridas e em escala de cinza, com as opções de varredura.

            \item[lib/nb.py] Mock up do decorador \mintinline{python3}{@numba.jit} \autocite{ref:numbajit}, para que a ferramenta funcione com ou sem a biblioteca Numba.

            \item[lib/horizontal.py] Implementação das varreduras horizontais: varredura unidirecional e a alternada.

            \item[lib/direcao.py] Controle de direção para varreduras não horizontais (espiral e curvas de Hilbert).

            \item[lib/espiral.py] Implementação da varredura em espiral.

            \item[lib/hilbert.py] Varredura seguindo uma curva de Hilbert.
        \end{description}

        \item[dists.py] Definição das distribuições de erro ao longo da operação de meios-tons.

        \item[inout.py] Funções que tratam da entrada e saída do programa, como leitura e escrita de arquivos de imagem e também da apresentação da imagem em uma janela gráfica.

        \item[tipos.py] Definição de alguns tipos para checagem estática com \texttt{mypy} \autocite{ref:mypy}.
    \end{description}

    Todas as figuras base utilizadas neste relatório podem ser encontradas na pasta \texttt{imagens} do código fonte, como descrito nos rótulos da \cref{fig:base}. Além disso, foi disponibilizado também um \textit{script} em \texttt{bash}, \texttt{run.sh}, que realiza todos os processamentos requeridos em cada uma das imagens na pasta.

\subsection{Execução} \label{sec:execucao}

    A execução deve ser feita através do interpretador de Python 3.7+. A única entrada obrigatória é o caminho para a imagem PNG que será processada. Ao final da execução, a imagem resultante será exibida na tela.

    % \begin{figure}[H]
    %     \centering
    %     \includegraphics[width=6cm]{resultados/execucao.png}

    %     \caption{Aplicação de pontilhado com a \texttt{baboon.png}.}
    %     \label{fig:execucao}
    % \end{figure}

    Os argumentos opicionais podem ser vistos com \mintinline{bash}{$ python3 main.py --help}. A mais importante das opções é \mintinline{text}{--output}, ou \mintinline{text}{-o}, que salva o resultado em um arquivo PNG em vez de exibir na tela. Se é desejável tanto a exibição da imagem quanto o salvamento no arquivo, o argumento \mintinline{text}{--force-show} ou \mintinline{text}{-f} pode ser usado. Também existe a opção \mintinline{bash}{--grayscale} ou \mintinline{bash}{-g} que faz o processamento em escala de cinza.

    As outras opções são referentes à técnica de pontilhado. A flag \mintinline{bash}{--varredura} ou \mintinline{bash}{-v} controla a forma de varrredura na imagem, como descrito na \cref{sec:varredura}. A distribuição de erros aplicada pode ser modificada com \mintinline{bash}{--destribuicao} ou \mintinline{bash}{-d} e aceita o nome de qualquer um dos idealizadores da distribuição.

    Por exemplo, o comando abaixo apresenta a \cref{fig:execucao} em uma nova janela gráfica.

    \begin{minted}{bash}
        $ python3 main.py imagens/baboon.png -g -d ninke
    \end{minted}

\subsection{A biblioteca Numba}

    A ferramenta foi desenvolvida com requerimento mínimo sendo apenas o NumPy. No entanto, se presente, a biblioteca Numba causa um grande diferencial no tempo de execução das técnicas de pontilhado (cerca de trinta vezes mais rápido na máquina de desenvolvimento). Por isso, a biblioteca é fortemente recomendada na execução do código.


    % \section{Implementação} \label{sec:impl}

\subsection{Técnicas de Limiarização}

    A limiarização aplicada aqui se baseia em encontrar um limiar $T$, escolhendo cada pixel como objeto se o valor dele for maior que $T$ ou como fundo, no caso contrário. Nesse trabalho, os pixels do objeto foram coloridos em preto e o fundo em branco.

    Os métodos globais em geral escolhem um mesmo limiar $T$ para toda a imagem, mas o método \mintinline{bash}{global} implementado aqui escolhe, por padrão, $T$ como o valor médio na imagem. Isso pode ser alterado com a \textit{flag} \mintinline{bash}{-T} na linha de comando, mudando o limiar global para o valor dado.

    Por outro lado, os métodos locais escolhem um limiar $T(x, y)$ para cada pixel $(x, y)$ aplicando uma função na vizinhança do pixel. Essa vizinhança é escolhida a partir de um raio $r = 5$, que pode ser alterado na linha de comando (opção \mintinline{bash}{-r}). Existe a opção também de alterar a medida de distância, usando vizinhança-4, com a \textit{flag} \mintinline{bash}{-v4}, em vez da vizinhança-8, que é o padrão. Para o método \mintinline{bash}{phansalkar}, foi necessário também normalizar a imagem para o intervalo $[0, 1]$ antes do processamento.

    Todos os métodos locais foram implementados em C, como discutido anteriormente, para uso com a função \mintinline{python3}{scipy.ndimage.generic_filter}. As bordas da imagem são extendidas com a opção \mintinline{python3}{"mirror"}, que reflete os pixels mais próximos da fronteira, mas sem repetir o pixel exatamente na fronteira.

\begingroup
\subsection{Imagens Base} \label{sec:imgbase}
\newcommand{\histograma}[2][width=\textwidth]{
    \begin{figure}[H]
        \centering
        \begin{subfigure}{0.35\textwidth}
            \centering
            \includegraphics[#1]{base/#2.png}
            \caption{Imagem.}
        \end{subfigure}%
        \begin{subfigure}{0.64\textwidth}
            \centering
            \import{../resultados/hist}{#2.pgf}
            \caption{Histograma.}
        \end{subfigure}

        \caption{\texttt{imagens/#2.pgm}}
        \label{fig:base:#2}
    \end{figure}
}

A seguir temos as sete imagens usadas como base de comparação dos métodos de limiarização. As imagens podem ser encontradas na pasta \mintinline{bash}{imagens} do código fonte, a partir do nome apresentado no rótulo de cada uma.

Para cada imagem também está apresentado o histograma dos pixels, mostrando a quantidade de vezes que cada valor de pixel aparece na imagem. Podemos ver de forma geral nos histogramas o quão escura uma imagem é em média e o quão distribuído são os valores. Isso pode ajudar a ver a clara separação no histograma entre objeto e fundo, como acontece \texttt{fiducial.pgm} (\cref{fig:base:fiducial}).

\histograma{baboon}

\histograma{fiducial}

\histograma{monarch}

\histograma{peppers}

\histograma{retina}

\histograma{sonnet}

\histograma{wedge}

\endgroup



    % \section{Resultados}
\captionsetup{justification=centering}
\newcommand{\metodo}[6][]{%
    \ifblank{#1}{%
        \begin{subfigure}{0.45\textwidth}
            \centering
            \ifblank{#4}{
                \includegraphics[width=0.9\textwidth]{#3/#2.png}
                \caption{~\texttt{#2.pgm} com #5.\\ $#6\%$ de pixels pretos.}
                \label{fig:#3:#2}
            }{
                \includegraphics[width=0.9\textwidth]{#3/#4.png}
                \caption{~\texttt{#2.pgm} com #5.\\ $#6\%$ de pixels pretos.}
                \label{fig:#3:#4}
            }
        \end{subfigure}%
    }{%
        \begin{subfigure}{0.9\textwidth}
            \centering
            \ifblank{#4}{
                \includegraphics[width=0.45\textwidth]{#3/#2.png}
                \caption{~\texttt{#2.pgm} com #5. $#6\%$ de pixels pretos.}
                \label{fig:#3:#2}
            }{
                \includegraphics[width=0.45\textwidth]{#3/#4.png}
                \caption{~\texttt{#2.pgm} com #5. $#6\%$ de pixels pretos.}
                \label{fig:#3:#4}
            }
        \end{subfigure}%
    }
}

\subsection{Método Global}

\begin{figure}[H]
    \centering
    \metodo[wide]{retina}{global}{}{$T = 128$}{15}\\[8pt]
    \metodo{wedge}{global}{100}{$T = 100$}{72}%
    \metodo{wedge}{global}{110}{$T = 110$}{49}

    \caption{Exemplos de aplicação do método \texttt{global}.}
    \label{fig:global}
\end{figure}

Esse é o método mais simples, definindo um limiar fixo $T$ para toda a imagem. Ele funciona razoavelmente bem para imagens com iluminação regular, como pode ser visto na \cref{fig:global:retina}.

Por outro lado, figuras com iluminação mais natural ou irregular podem ter uma grande variação da intensidade dos pixels ao longo do objeto. Isso faz com a separação do objeto de forma global seja praticamente impossível, levando a problemas como os das figuras \ref{fig:global:wedge100} e \ref{fig:global:wedge110}.


\end{document}
