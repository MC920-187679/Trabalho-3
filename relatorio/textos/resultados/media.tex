\section{Método da Média}

O método da média define o limiar apenas como a média da vizinhança, isto é, $T(x, y) = \mu(x, y)$. Isso funciona razoavelmente bem para imagens onde o objeto tem apenas uma cor, como é caso das figuras \ref{fig:media:fiducial} e \ref{fig:media:wedge}.

No entanto, imagens pequenas como \texttt{retina.pgm}, onde o raio não pode ser muito grande, o método acaba deixando resultados pouco interessantes.

\begin{figure}[H]
    \centering
    \metodo{fiducial}{media}{}{$r = 50$}{64}%
    \metodo{wedge}{media}{}{$r = 50$}{52}\\[8pt]
    \metodo[wide]{retina}{media}{}{$r = 30$}{43}

    \caption{Exemplos de aplicação do método \texttt{media}.}
    \label{fig:media}
\end{figure}
