\subsection{Método de Phansalkar, More e Sabale}

O método desenvolvido por Phansalkar et al. teve forte influência do método de Sauvola, apresentado na seção anterior. A principal diferença é um fator exponencial na fórmula, o que faz com que o limiar não caia muito rapidamente em regiões de baixa intensidade. Essa técnica foi pensanda com foco em imagens celulares, com tratamento para cores e usando a intensidade normalizada em $[0, 1]$. Nessa técnica, a fórmula do limiar é dada por:
\[
    T(x, y) = \mu(x, y) \left[1 + p e^{-q \mu(x, y)} + k \left(\frac{\sigma(x, y)}{R} - 1\right)\right]
\]

Podemos ver que o método consegue reduzir a força da borda da retina na \cref{fig:phansalkar:retina}, mas mantendo os vasos sanguíneos. Além disso, na \texttt{monarch.pgm}, a borboleta consegue ser mantida removendo ainda boa parte da folhagem, mas alguns detalhes são perdido. Esse método consegue também começar a marcar o centro do objeto na \texttt{wedge.pgm} (\cref{fig:phansalkar:wedge}).

\begin{figure}[H]
    \centering
    \metodo[wide]{retina}{phansalkar}{}{$r = 5$, $k = 0.2$, $R = 4$, $p = 0.2$ e $q = 0.5$}{5}\\[8pt]
    \metodo{monarch}{phansalkar}{}{$r = 10$, $k = 0.9$, $R = 0.7$, $p = 5$ e $q = 10$}{95}%
    \metodo{wedge}{phansalkar}{}{$r = 10$, $k = 0.1$, $R = 0.9$, $p = 12$ e $q = 16$}{83}

    \caption{Exemplos de aplicação do método \texttt{phansalkar}.}
    \label{fig:phansalkar}
    \vskip-1.2em
\end{figure}