\section{Introdução} \label{sec:intro}

Este trabalho teve como objetivo a implementação de técnicas de pontilhado com difusão de erros. A ideia é reduzir a quantidade de cores tentando manter a imagem o mais próximo possível da original. Isso é feito reduzindo a intesidade para seu limite mais próximo, ao longo de um caminho pela imagem, mas aplicando uma distribuição de erros na vizinhança do pixel.

A \cref{fig:base} apresenta as imagens base deste trabalho, usadas para análise e discussão das formas diferentes de aplicar o pontilhado. As figuras \ref{fig:baboon} e \ref{fig:peppers} são $512 \times 512$, enquanto a \cref{fig:monalisa} é $256 \times 256$ e a \cref{fig:watch} é $1024 \times 768$. Ao todo serão aplicadas 6 distribuições de erro com 4 curvas diferentes em cada figura. As distribuições de erros estão apresentadas na \cref{sec:distribuicoes}, equanto as curvas de varrimento podem ser encontradas na \cref{sec:varredura}.

% \begin{figure}[H]
%     \centering
%     \newcommand{\histograma}[2][width=\textwidth]{
    \begin{figure}[H]
        \centering
        \begin{subfigure}{0.35\textwidth}
            \centering
            \includegraphics[#1]{base/#2.png}
            \caption{Imagem.}
        \end{subfigure}%
        \begin{subfigure}{0.64\textwidth}
            \centering
            \import{../resultados/hist}{#2.pgf}
            \caption{Histograma.}
        \end{subfigure}

        \caption{\texttt{imagens/#2.pgm}}
        \label{fig:base:#2}
    \end{figure}
}

%     \caption{Imagens base da comparação dos filtros.}
%     \label{fig:base}
% \end{figure}
