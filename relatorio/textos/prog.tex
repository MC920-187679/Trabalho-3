\section{O Programa}

Além das bibliotecas padrão de Python, foram utilizados os pacotes NumPy \autocite{ref:numpy} e, de forma opcional, Numba \autocite{ref:numba}.

\subsection{Código Fonte}

    O programa foi desenvolvido em Python 3.8, mas deveria funcionar com a versões 3.7 e 3.9 também. Além disso, o código fonte foi separado nos seguintes arquivos:

    \begin{description}
        \item[main.py] É o corpo do programa, resposável por processar os comandos e as opções da linha de comando.

        \item[lib] Pacote interno com as operações de pontilhado.

        \begin{description}
            \item[lib/\_\_init\_\_.py] Funções gerais de aplicação do pontilhado em imagens coloridas e em escala de cinza, com as opções de varredura.

            \item[lib/nb.py] Mock up do decorador \mintinline{python3}{@numba.jit} \autocite{ref:numbajit}, para que a ferramenta funcione com ou sem a biblioteca Numba.

            \item[lib/horizontal.py] Implementação das varreduras horizontais: varredura unidirecional e a alternada.

            \item[lib/direcao.py] Controle de direção para varreduras não horizontais (espiral e curvas de Hilbert).

            \item[lib/espiral.py] Implementação da varredura em espiral.

            \item[lib/hilbert.py] Varredura seguindo uma curva de Hilbert.
        \end{description}

        \item[dists.py] Definição das distribuições de erro ao longo da operação de meios-tons.

        \item[inout.py] Funções que tratam da entrada e saída do programa, como leitura e escrita de arquivos de imagem e também da apresentação da imagem em uma janela gráfica.

        \item[tipos.py] Definição de alguns tipos para checagem estática com \texttt{mypy} \autocite{ref:mypy}.
    \end{description}

    Todas as figuras base utilizadas neste relatório podem ser encontradas na pasta \texttt{imagens} do código fonte, como descrito nos rótulos da \cref{fig:base}. Além disso, foi disponibilizado também um \textit{script} em \texttt{bash}, \texttt{run.sh}, que realiza todos os processamentos requeridos em cada uma das imagens na pasta.

\subsection{Execução} \label{sec:execucao}

    A execução deve ser feita através do interpretador de Python 3.7+. A única entrada obrigatória é o caminho para a imagem PNG que será processada. Ao final da execução, a imagem resultante será exibida na tela.

    % \begin{figure}[H]
    %     \centering
    %     \includegraphics[width=6cm]{resultados/execucao.png}

    %     \caption{Aplicação de pontilhado com a \texttt{baboon.png}.}
    %     \label{fig:execucao}
    % \end{figure}

    Os argumentos opicionais podem ser vistos com \mintinline{bash}{$ python3 main.py --help}. A mais importante das opções é \mintinline{text}{--output}, ou \mintinline{text}{-o}, que salva o resultado em um arquivo PNG em vez de exibir na tela. Se é desejável tanto a exibição da imagem quanto o salvamento no arquivo, o argumento \mintinline{text}{--force-show} ou \mintinline{text}{-f} pode ser usado. Também existe a opção \mintinline{bash}{--grayscale} ou \mintinline{bash}{-g} que faz o processamento em escala de cinza.

    As outras opções são referentes à técnica de pontilhado. A flag \mintinline{bash}{--varredura} ou \mintinline{bash}{-v} controla a forma de varrredura na imagem, como descrito na \cref{sec:varredura}. A distribuição de erros aplicada pode ser modificada com \mintinline{bash}{--destribuicao} ou \mintinline{bash}{-d} e aceita o nome de qualquer um dos idealizadores da distribuição.

    Por exemplo, o comando abaixo apresenta a \cref{fig:execucao} em uma nova janela gráfica.

    \begin{minted}{bash}
        $ python3 main.py imagens/baboon.png -g -d ninke
    \end{minted}

\subsection{A biblioteca Numba}

    A ferramenta foi desenvolvida com requerimento mínimo sendo apenas o NumPy. No entanto, se presente, a biblioteca Numba causa um grande diferencial no tempo de execução das técnicas de pontilhado (cerca de trinta vezes mais rápido na máquina de desenvolvimento). Por isso, a biblioteca é fortemente recomendada na execução do código.
