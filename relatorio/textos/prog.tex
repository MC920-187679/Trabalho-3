\section{O Programa}

O corpo do programa foi desenvolvido em Python 3.6+, utilizando a biblioteca padrão em conjunto com as bibliotecas OpenCV, para leitura e escrita das imagens, NumPy e SciPy, para as técnicas de limiarização local.

Em especial, a limiarização local foi implementada com a função \mintinline{python3}{generic_filter} \autocite{ref:genericfilter} do SciPy. Para acelerar um pouco o processamento, as funções de cálculo do limiar que são passadas como argumento da \mintinline{python3}{generic_filter} foram implementadas em C (padrão GNU11) e requerem o compilador GCC 4.7+ na primeira execução do programa.

\subsection{Código Fonte}

    O código fonte foi separado nos seguintes arquivos, dentro da pasta \texttt{limiarizacao}:

    \begin{description}
        \item[\_\_main\_\_.py] Processamento os comandos e as opções da linha de comando.

        \item[metodos] Pacote interno com as operações de limiarização.

        \begin{description}
            \item[metodos/\_\_init\_\_.py] Classes e objetos para as técnicas de limiarização e tratamento dos parâmetros de cada técnica, pelos argumentos da linha de comando.

            \item[metodos/locais.py] Wrapper para a chamada das funções em C.

            \item[metodos/locais.c] Funções de cálculo do limiar da vizinhança do pixel.

            \item[metodos/ops.c] Operações de mínimo, máximo, média, desvio padrão e mediana.
        \end{description}

        \item[inout.py] Tratamento de entrada e saída do programa.

        \item[tipos.py] Tipos para checagem estática com MyPy.
    \end{description}

    Além disso, existem outros arquivos junto com o código fonte. A pasta \texttt{imagens} contém as figuras base utilizadas neste relatório, apresentadas na \red{SECIMG}. \red{FOI ??} disponibilizado também um \textit{script} em \texttt{bash}, \texttt{run.sh}, que realiza todos os processamentos apresentados no relatório \red{RESULTADOS?}.

\subsection{Execução} \label{sec:execucao}

    A execução deve ser feita através do interpretador de Python, com as seguintes entradas obrigatórias: o caminho para a imagem PGM que será processada e o método de limiarização a ser aplicado. Ao final da execução, a imagem resultante será exibida na tela.

    Os argumentos opicionais podem ser vistos com \mintinline{bash}{$ python3 limiarizacao --help}. A primeira opção é \mintinline{text}{--output}, ou \mintinline{text}{-o}, que salva o resultado em um arquivo PNG ou PGM em vez de exibir na tela. Se é desejável tanto a exibição da imagem quanto o salvamento no arquivo, o argumento \mintinline{text}{--force-show} ou \mintinline{text}{-f} pode ser usado.

    O método de limiarização deve vir após essas opções e pode ser uma das seguintes chaves: \mintinline{bash}{global}, \mintinline{bash}{bernsen}, \mintinline{bash}{niblack}, \mintinline{bash}{sauvola}, \mintinline{bash}{phansalkar}, \mintinline{bash}{contraste}, \mintinline{bash}{media} ou \mintinline{bash}{mediana}. Os parâmetros de cada método pode ser visto com \mintinline{bash}{$ python3 limiarizacao METODO --help}, onde \mintinline{bash}{METODO} é a técnica desejada.

    Por exemplo, o comando abaixo apresenta executa o método de Bernsen com vizinhança de raio 6 em uma imagem \mintinline{bash}{entrada.pgm}, salvando depois o resultado em \mintinline{bash}{saida.png}.

    \begin{minted}{bash}
        $ python3 limiarizacao entrada.pgm -o saida.png bernsen -r 6
    \end{minted}
